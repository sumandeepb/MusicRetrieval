\documentclass{llncs}

\begin{document}

\title{Content Based Retrieval of Music Videos}

\author{Sumandeep Banerjee}

\institute {German Research Center for Artificial Intelligence\\
Erwin-Schroedinger-Strasse 57, 67663 Kaiserslautern, Germany \\
\email{sumandeep.banerjee@gmail.com}}

\maketitle

\begin{abstract}
In Music Videos most of the relevant information exists in the audio
track. We present a scheme of music video retrieval involving the
feature extraction on the audio track, clustering the feature
vectors, and an algorithm of querying, which returns the closest
matching segment of a song, rather than the whole song itself.

\emph{Keywords} - Music information retrieval, matching.
\end{abstract}

\section{Introduction}
Growing digital media collections have created the demand for more
improved retrieval mechanisms. Significant amount of work has been
done on music information retrieval as well as on video retrieval.
But the problem of music video retrieval, where the relevant
information is in the audio track, rather than the video has not yet
been properly explored.

In this work, we present a scheme of retrieving songs from an entire
collection of music videos. The query matching algorithm is such
that, it is able to return the closest matching segment within the
song, without significant computations, or searching using
similarity measure throughout the entire length of the song.

In Section 2 we specify the feature extraction methodology employed.
In Section 3 we discuss clustering of the data. The query matching
algorithm is presented in Section 4. Section 5 presents an analysis
of the computational complexity of the query algorithm. Section 6
concludes the paper.

\subsection{Previous Work}
Most of the work in Music Information Retrieval follow a common
structure of feature extraction, followed by classification /
indexing and finally a similarity measure with varying set of
parameters. A detailed analysis of such methods, have been presented
in ~\cite{PACHET04}. Some of the methods are inclined towards
classification ~\cite{TZAN02}, ~\cite{MANDEL05}, while others,
~\cite{LOGAN01} concentrate on playlist generation based on
similarity measures.

For feature extraction in most of the methods, as is in the present
work, MFCCs are used. ~\cite{TZAN02} proposes other additional
parameters such as Spectral Centroid, Spectral Rolloff, Spectral
Flux and Time Domain Zero Crossings. But these seem to have more
usage in classification, rather that retrieval, and were not
considered in the present work.

The statistical modeling of the MFCC distribution ~\cite{PACHET04}
is computed using a variety of ways. Some use K-means
~\cite{LOGAN01}, ~\cite{LOGAN03}, while others have employed
Gaussian Mixture Models (GMMs) ~\cite{LOGAN03}, ~\cite{PACHET02},
~\cite{KULESH03}.

Finally, the distance measure is done using any of Euclidean, Earth
Mover's Distance ~\cite{LOGAN01}, KL divergence ~\cite{MANDEL05},
Mahalanobis distance ~\cite{MANDEL05}.

Almost all of the various works mentioned above are either
interested in classifying the songs into music genres
~\cite{TZAN02}, ~\cite{MANDEL05}, or on song recomendation
~\cite{LOGAN01}, but none does mention ways of determining matching
segments of song, rather than the whole song itself. This problem is
addressed in the present work.

\section{Feature Extraction}
In a music video almost all relevant information is contained in the
audio track. Hence, the feature extraction is performed on the audio
track only. The audio track extraction is done using Media
Coder~\cite{MC}. Mel Frequency Cepstral Coefficients ~\cite{JUAN93}
have long been used as feature vectors in speech recognition. Logan
~\cite{LOGAN00} showed that the assumptions in computing MFCCs are
valid for music modeling. Other works ~\cite{LOGAN01}, ~\cite
{TZAN02}, ~\cite{MANDEL05}, have successfully employed MFCCs for
music information retrieval.

MFCC feature extraction is motivated by perceptual and computational
considerations ~\cite{LOGAN00}.The first step is to take short
snippets of the audio signal, by applying a windowing function at
fixed intervals. Each frame thus generated is a segment of the
signal which is statistically stationary. The window function
removes edge effects.

\begin{figure}[ht]
\setlength{\unitlength}{1cm}
\begin{picture}(11.5,9.2)(0,0)

\put(0,1.0){\line(1, 0){5}} \put(0,2.0){\line(1, 0){5}}
\put(0,2.5){\line(1, 0){5}} \put(0,3.5){\line(1, 0){5}}
\put(0,4.0){\line(1, 0){5}} \put(0,5.0){\line(1, 0){5}}
\put(0,5.5){\line(1, 0){5}} \put(0,6.5){\line(1, 0){5}}
\put(0,7.0){\line(1, 0){5}} \put(0,8.0){\line(1, 0){5}}

\put(0,1.0){\line(0, 1){1}} \put(5,1.0){\line(0, 1){1}}
\put(0,2.5){\line(0, 1){1}} \put(5,2.5){\line(0, 1){1}}
\put(0,4.0){\line(0, 1){1}} \put(5,4.0){\line(0, 1){1}}
\put(0,5.5){\line(0, 1){1}} \put(5,5.5){\line(0, 1){1}}
\put(0,7.0){\line(0, 1){1}} \put(5,7.0){\line(0, 1){1}}

\put(0.85, 0.0){$MFCC Feature Vector$}
\put(0.35, 1.5){$Discrete Cosine Transform$}
\put(1.35, 3.0){$Mel smoothing$}
\put(0.35, 4.5){$Log of Amplitude Spectrum$}
\put(0.35, 6.0){$Discrete Fourier Transform$}
\put(1.00, 7.5){$Window Function$}
\put(1.45, 8.8){$Input Signal$}

\put(2.5, 1.0){\vector(0, -1){0.5}}
\put(2.5, 2.5){\vector(0, -1){0.5}}
\put(2.5, 4.0){\vector(0, -1){0.5}}
\put(2.5, 5.5){\vector(0, -1){0.5}}
\put(2.5, 7.0){\vector(0, -1){0.5}}
\put(2.5, 8.5){\vector(0, -1){0.5}}

\put(5.75,1){\line(1, 0){5.75}}
\put(5.75,8){\line(1, 0){5.75}}
\put(5.75,1){\line(0, 1){7.0}}
\put(9.0,1){\line(0, 1){7.0}}
\put(11.5,1){\line(0, 1){7.0}}

\put(6.0, 7.5){$Input Signal$}
\put(6.0, 5.5){$Window Function$}
\put(6.0, 4.5){$Window Size$}
\put(6.0, 3.5){$Hop Size$}
\put(6.0, 2.5){$FFT Size$}
\put(6.0, 1.5){$MFCC Features$}

\put(9.25, 7.5){$44.1 KHz$} \put(9.25, 6.5){$16 bit Mono$}
\put(9.25, 5.5){$Hamming$} \put(9.25, 4.5){$11ms$} \put(9.25,
3.5){$6ms$} \put(9.25, 2.5){$512$} \put(9.25, 1.5){$13$}

\end{picture}
\caption{MFCC Feature Vector (a) Computational Steps (b)
Parameters.\label{MFCC}}
\end{figure}

In the next step Discrete Fourier Transform (DFT) is applied on each
frame. The logarithm of the amplitude signal is retained. Perceptual
studies have shown that the amplitude is more important than the
phase, and the perceived loudness of a signal is approximately
logarithmic.The next step is to smooth the spectrum, and emphasize
perceptually meaningful frequency components. The lower frequencies
were found to be more perceptually important then the higher
frequencies. This step is known as mel smoothing. The mel spectral
vectors are highly correlated, thus to decorrelate the components,
Discrete Cosine Transform (DCT) is applied. The transform domain
output is our feature vector.

 The input signal is split into frames of 11ms each, with an overlap
of 6ms. The feature vector is represented by 13 MFCC coefficients,
thereby generating feature vectors of 13 dimensions representing
11ms of signal. Such high resolution in time domain is generally not
required. The MFCC feature vectors are further down-sampled to
represent 250ms of data. This greatly reduces the number of feature
vectors, without significant compromise in the retention of
information about the signal.

\section{Clustering}
Querying by similarity measure over all the feature vectors in the
database would be extremely computationally expensive. Thus,
clustering is performed on the entire feature vectors using K-Means
clustering. The number of clusters is chosen as $K = \sqrt{N}$,
where $N$ is the number of feature vectors in the song database
~\cite{MAULIK02}, ~\cite{KASTOR03}. Alternatively, the number of
clusters can also be chosen by the method presented by Pelleg and
Moore in ~\cite{PELLEG00}, but have been not used in the
implementation, as this is beyond the scope of the present work. All
the feature vectors are clustered and cluster centers are stored to
be used for querying, thereby reducing the search complexity for a
query from $O(N)$ to $O(K)$. This greatly improves the search
efficiency.

\section{Query Matching Algorithm}
The query matching algorithm proposed is quite unique. The various
steps involved in the analysis of the query and generation of
matching segments of song, is explained one by one in this section.

\begin{figure}
\texttt{Input : Query i.e., audio signal in time domain
\\Step 1: Compute MFCCsfor the query
\\Step 2: Do for each MFCC frame
\\    a. Find nearest cluster center
\\    b. Cluster members are potential hits
\\    c. Collect all cluster members
\\Step 3: Group the hits (from step 2) by their song ID
\\Step 4: Do for each song
\\    a. Identify contiguous / semi-contiguous segments of the song
\\    b. Apply a smoothing function to the segments
\\Step 5: Rank the identified segments using the Hamming distance
between the original and the smoothed version of the segments
\\Output: Ranked list of song segments matching the query}
\caption{QueryDB - Query Matching Algorithm\label{QUERYDB}}
\end{figure}

\subsection{Feature Extraction}
The query audio signal is split into frames as explained above in
the section:feature extraction. Each frame represents 250ms of
temporal signal, and is denoted by a 13 dimensional MFCC feature
vector.

\subsection{Winner List}
Each MFCC feature vector generated from the query is compared with
all the cluster centers, and the nearest cluster center is
determined using $L_2$ Norm (Euclidean). We get one winner cluster
center per frame of the query audio signal.

The collection of cluster members (MFCC feature vector of frame) of
the entire list of winner cluster centers contains all the frames
having a possible match in the query snippet.

\subsection{Matching Segment}
The collection of frames in the winner list are grouped by the songs
they belong to. The frames of a song which appear in the list are
denoted by '1' and frames which do not appear are denoted by '0'.
The sequence of 1's and 0's are sorted according to their temporal
occurrence in the song. This sequence is called frame sequence of
the song.

The next step is to identify contiguous / semi-contiguous sequences
of 1's appearing in the frame sequence of the song, which may be
deduced as a possible matching segment for the query.

\subsection{Ranking of Matching Segments}
Each contiguous sequence of 1's in the frame sequence is regarded as
a matching song segment for the query snippet. In some sequences,
there may be a few outliers of 0's in between long sequences of 1's.
These occur when in some frame sequences some matches are missed due
to noise etc. Also there could be solitary 1's giving false
impression of a segment. These outliers break the contiguous nature
of the frame sequence. To improve the performance of the segment
matching, it is required that these outliers must be removed.

The smoothing of the frame sequence, to get rid of outliers, is done
by applying autoregressive low-pass filter (ARLPF). The ARLPF masks
small number of outlying 1's and 0's in the sequence, to produce
better contiguous matching segments within each song. We have used a
triangular ARLPF of length 11.

The rank of a matching segment is determined by measuring the degree
of smoothness i.e. by the extent of it's contiguous nature. The
matching segment before being smoothed by ARLPF is compared with the
smoothed segment. The hamming distance between the two gives a
numerical idea of it's smoothness. The hamming distance per frame,
gives a normalized rank value of the smoothness of the matching
segment. The lower the rank value, the segment is considered better
matching.

The process of identification and ranking of matching segments is
performed for each song identified in the hit list. The result is
displayed in the increasing order of the rank values for each
matching segment. The result specifies, the start and end times of
the matching segment, and the song in which it appears.

\section{Complexity Analysis}
I this section we will present an analysis of the computational
complexity of the proposed query matching algorithm.  First we will
begin with the definition of the parameters involved. \\\(N
:\textrm{number of frames / feature vectors in the database}\) \\\(K
:\textrm{number of clusters}\) \\\(D :\textrm{dimention of feature
vector}\) \\\(n :\textrm{number of frames in the query snippet}\)

Some properties of the parameters worth noting are, $n << N$, $K =
\sqrt{N}$, and $D = 13-40$.

The complexity of computing the list of winner cluster centers is
$O(nKD)$. As $K=\sqrt{N}$, the algorithm is sub-linear with respect
to the size of the database ($N$). In the worst case the number of
winner cluster centers is $n$ i.e. each frame of the query belongs
to a different cluster. The winner cluster center list is sorted for
easier further processing, using Quick Sort, which takes
$O(n{\log}n)$ time.

The generation of the winner list requires a single iteration over
the entire collection of cluster members of the winner clusters. The
average number of members in each cluster is $\frac{N}{K}$. Let the
number of winner clusters be $w = \frac{nN}{K}$. Thus, hit list
takes $O(w)$ amount of time to be prepared.

Let's assume that the number of frames in a song identified in the
winner list is $p$, where $p > n$ and $p << N$. The frame sequence
can be prepared in time $O(p)$. The smoothing of the frame sequence
using ARLPF takes $O(pg)$, where $g$ is the order of the ARLPF.

\section{Results and Conclusion}
For testing purposes, a collection of 65 most popular video songs of
2005 were used. The total number of MFCC feature vector frames ($N$)
for the database was 103,064. Each feature vector represented 250ms
of audio signal. The number of clusters were taken to be $\sqrt{N}$
as 321.

\begin{table}[ht]
\begin{tabular}{|r|p{2cm}|p{2cm}|l|r|r|}
\hline Rank & Normalized Hamming Distance & Matching segment length (Sec) &
Song ID & Start position (Sec) & End position (Sec)
\\\hline
   1&  6&    42& 37 Heartb &   14  &  56
\\ 2& 16&    71& 07 Smile. &   37  & 108
\\ 3& 16&    34& 07 Smile. &  199  & 234
\\ 4& 19&    24& 18 Leaf H &    6  &  30
\\ 5& 19&    17& 21 Us.wav &   82  &  99
\\ 6& 19&    27& 36 Breath &  193  & 220
\\ 7& 19&    17& 63 New Sl &  406  & 423
\\ 8& 21&    42& 25 Your E &   41  &  83
\\ 9& 21&    27& 36 Breath &  162  & 190
\\10& 21&    28& 49 Going  &  253  & 282
\\11& 23&    19& 10b Hitch &  270  & 289
\\12& 23&    17& 53 Mushab &  257  & 274
\\13& 24&    36& 07 Smile. &  162  & 198
\\14& 26&    18& 32 Walk A &  233  & 252
\\15& 31&    21& 37 Heartb &  130  & 151
\\\hline
\end{tabular}
\caption{Query result for a typical query of 15 second snippet from
song 37. Query time was 0.968 seconds\label{result}}
\end{table}

A typical query result is given in Table \ref{result}. For most of
the test runs, the top ranked result, came from the song from which
the query snippet was extracted. The segment of the result
determined by the algorithm, contained the snippet along with the
portion of the song most relevant to the timbral texture of the
query.

We can see from the discussion in the previous section that
complexity of all the steps in the algorithm are sub-linear with
respect to the size of the database ($N$), and most are linear with
respect to the query size ($n$). So, as our database is increased,
there won't be significant increase in the querying time. This is
quite good considering the fact, that we are not only choosing the
best matching song, but also able to predict the position of
occurrence of the query within the song.

There is lot of scope for further improvement, in areas like
determining the number of clusters ($K$) ~\cite{PELLEG00}, which
greatly affects the complexity of almost all the steps. The ARLPF is
central to the correctness / relevance of the result displayed.
Varying the order and shape of the ARLPF could have significant
effect on the results. Further research could be directed to explore
various possibilities in these areas.

\bibliographystyle{splncs}

\begin{thebibliography}{99}

\bibitem{PACHET04}
J.J.Aucouturier and F.Pachet: Improving timbre similarity: How high
is the sky? Journal of Negative Results in Speech and Audio
Sciences, 1(1), (2004).

\bibitem{TZAN02}
G.Tzanetakis and P.Cook: Musical genre classification of audio
signal. IEEE Transactions on Speech and Audio Processing, 10(5),
(July 2002).

\bibitem{MANDEL05}
M.Mandel and D.Ellis: Song-level features and support vector
machines for music classification. Proceedings of the Sixth
International Conference on Music Information Retrieval, (2005).

\bibitem{LOGAN03}
A.Berenzweig, B.Logan, D.P.W.Ellis, and B.Whitman: Proceedings of
the Fourth International Conference on Music Information Retrieval,
(2003).

\bibitem{LOGAN00}
B.Logan: Mel frequency cepstral coefficients for music modelling.
Proceedings of the First International Symposium on Music
Information Retrieval, (2000).

\bibitem{LOGAN01}
B.Logan and A.Salomon: A music similarity function based on signal
analysis. Proceedings of the IEEE International Conference on
Multimedia and Expo, (2001).

\bibitem{KULESH03}
V. Kulesh, I. Sethi, and P.V: Indexing and retrieval of music via
gaussian mixture models. Proceedings of the 3$^{rd}$ International
Workshop on Content-Based Multimedia Indexing, (2003).

\bibitem{PACHET02}
J.J.Aucouturier and F.Pachet: Music similarity measures: What's the
use? Proceedings of the 3$^{rd}$ International Conference on Music
Information Retrieval, (2002).

\bibitem{JUAN93}
L.R.Rabiner and B.H.Juang: Fundamentals of Speech Recognition.
Prentice-Hall, (1993).

\bibitem{PELLEG00}
D.Pelleg and A.Moore: X-means: Extending k-means with efficient
estimation of the number of clusters. Proceedings of the Seventeenth
International Conference on Machine Learning, (2000).

\bibitem{MAULIK02}
U.Maulik and S.Bandyopadhyay: Performance Evaluation of Some
Clustering Algorithms and Validity Indices. IEEE Transactions on
PAMI, 24(12):1650�1654, (2002).

\bibitem{KASTOR03}
T.Kaster, V.Wendt and G.Sagerer: Comparing Clustering Methods for
Database Categorization in Image Retrieval. Proceedings, Lecture
Notes in Computer Science,  Springer-Verlag (2781):228�235, (2003).

\bibitem{MC}
Media Coder: http://www.rarewares.org/mediacoder/

\end{thebibliography}

\end{document}
